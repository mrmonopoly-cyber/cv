\documentclass[11pt,a4paper,sans]{moderncv}

% ModernCV themes
\moderncvstyle{banking}
\moderncvcolor{blue}

% Encoding
\usepackage[utf8]{inputenc}
\usepackage[scale=0.85]{geometry}

% Personal data
\name{[Alberto]}{[Damo]}
\title{Kernel \& Systems Software Developer}
\address{[Oderzo, Italy]}{}
\email{[alberto.damo@proton.me]}
\homepage{github.com/mrmonopoly-cyber}
\social[linkedin]{alberto-damo-62b81b23a}
\extrainfo{Master’s Student in Computer Science}

\begin{document}
\makecvtitle

%----------------------------------------------------------------------------------------
% PROFILE
%----------------------------------------------------------------------------------------
\section{Profile}
Systems and embedded software developer specializing in \textbf{low-level C programming}, \textbf{operating system internals}, and \textbf{bare-metal development}. 
Currently developing a \textbf{UEFI-compliant OS from scratch}, including a custom bootloader, kernel, and disk image generation utility for QEMU (Tianocore). 
Passionate about Linux, kernel programming, and open-source collaboration. 
Seeking to contribute to IBM’s Linux kernel and virtualization initiatives while deepening expertise in systems architecture and performance.

%----------------------------------------------------------------------------------------
% TECHNICAL SKILLS
%----------------------------------------------------------------------------------------
\section{Technical Skills}
\cvitem{Programming}{C (expert), Assembly (x86\_64 / TriCore basics), Shell scripting}
\cvitem{Systems}{Linux internals, bootloaders, UEFI, memory management, kernel architecture}
\cvitem{Embedded}{AURIX TC375 (TriCore), tricore-elf-gcc, linker scripts, atomic operations, GPIO \& CAN control}
\cvitem{Tools}{Git, GCC, GDB, QEMU, OVMF (Tianocore), libgpiod, SocketCAN, CI/CD}
\cvitem{Concepts}{Concurrency, interrupts, synchronization, atomic operations, virtualization fundamentals}
\cvitem{Open Source}{GitHub project maintainer; enforces branch protection and CI for integrity}

%----------------------------------------------------------------------------------------
% EDUCATION
%----------------------------------------------------------------------------------------
\section{Education}
\cventry{[Expected Graduation Year]}{Master’s Degree in Computer Engineering}{[Your University Name]}{}{}{
Focus areas: Operating Systems, Embedded Systems, Computer Architecture.}

%----------------------------------------------------------------------------------------
% PROJECTS
%----------------------------------------------------------------------------------------
\section{Selected Projects}

\cventry{2025--Present}{\textbf{Custom OS}}{UEFI-Compliant Operating System from Scratch}{}{}{
\begin{itemize}%
  \item Designed and implemented a \textbf{UEFI bootloader} and \textbf{monolithic kernel} entirely in C and Assembly.
  \item Built a UEFI-compliant \textbf{disk image generation utility} compatible with QEMU (OVMF / Tianocore).
  \item Developed early boot stages including \textbf{memory initialization}, console I/O, and process management foundations.
\end{itemize}}

\cventry{2023--Present}{\textbf{ControlUnitLogicOperator}}{Linux-Based Multi-Component System Simulator}{}{}{
\begin{itemize}%
  \item Built a modular simulation environment of 46 emulated components communicating via shared CAN buses and GPIOs.
  \item Implemented a \textbf{runtime trace stack} for error diagnostics using \_\_FILE\_\_ and \_\_LINE\_\_.
  \item Designed a \textbf{Safety Critical System (SCS)} with atomic synchronization across threads.
\end{itemize}}

\cventry{2023--Present}{\textbf{DPS (Debug Peripheral System)}}{Command-Line Debugging Interface}{}{}{
\begin{itemize}%
  \item Created a \textbf{CLI-based debugging tool} supporting live monitoring and introspection of system state.
  \item Provides modular runtime telemetry for GPIO and CAN buses.
\end{itemize}}

\cventry{2024}{\textbf{AURIX TC375 Firmware}}{Bare-Metal Safety-Critical Software}{}{}{
\begin{itemize}%
  \item Developed firmware for Infineon AURIX TC375 using custom linker scripts and atomic synchronization.
  \item Designed deterministic exception-safe handling without standard libraries.
\end{itemize}}

%----------------------------------------------------------------------------------------
% EXPERIENCE
%----------------------------------------------------------------------------------------
\section{Experience}
\cventry{2023--Present}{Independent Systems Developer}{}{Self-directed}{}{
\begin{itemize}%
  \item Designed architectures for both embedded and OS-level systems with a focus on portability and maintainability.
  \item Built custom debugging infrastructure with traceable runtime analysis.
  \item Mentored peers on sustainable system design and CI/CD integration.
\end{itemize}}

%----------------------------------------------------------------------------------------
% ACHIEVEMENTS
%----------------------------------------------------------------------------------------
\section{Achievements}
\cvitem{}{- Currently building a fully bootable UEFI-compliant OS from scratch.}
\cvitem{}{- Published open-source systems demonstrating atomic synchronization and debugging architecture.}
\cvitem{}{- Built a custom 32-bit Linux environment for emulation and testing.}

%----------------------------------------------------------------------------------------
% ADDITIONAL INFO
%----------------------------------------------------------------------------------------
\section{Additional Information}
\cvitem{Languages}{Fluent in English}
\cvitem{Interests}{Kernel engineering, virtualization, systems security, open-source development}
\cvitem{Goal}{Contribute to IBM’s Linux kernel and confidential container initiatives}

\end{document}

